\addchapter{Growth and Characterization of Weyl Semimetals}
\label{growth_of_wsm}

Weyl semimetals in this study were  grown using Molecular beam epitaxy (MBE) on c-plane 
of MgO substrate from Crystec. The substrate was cleaned using methanol
as described in \tref{wsm-substrate} and loaded in \tamarind where it was degassed 
at \temperature{550} for about 1 hour. Nb(Ta) and P are co evaoparoated 
from e-beam and K-cells respectively such that ther flx ratio is 1:20. The growth is 
monitored continuously using RHEED and the stoichiometry and morphology is assessed 
intermittently using XPS. Once the sample is taken out of the chamber its structure is 
assessed using XRD.

\section{Choice of Substrate}
\tlabel{wsm-substrate}

\begin{table}
    \tlabel{wsm-lattice-table}
    \caption {
            Lattice parameters of different materials in the plane of interest
        }
    \centering
    \begin{tabular}{lccccc}
        \toprule
        Material & NbP[100] & TaP[100] & Nb[110] & Ta[110] & MgO[110]\\
        \midrule
        Lattice \\Parameter $\left(\angstrom\right)$ & 3.33 & 3.34 & 3.30 & 3.30 & 3.03\\
        \bottomrule
    \end{tabular}
\end{table}


\begin{figure}
   
    \centering
    \subfloat[]{
        \ig{0.31}{structure/100-MgO.png}
        \tlabel{structure-MgO-100}
    }
    \subfloat[]{
        \ig{0.31}{structure/100-Nb-Ta.png}
        \tlabel{structure-Nb_Ta-100}
    }
    \subfloat[]{
        \ig{0.31}{structure/100-NbP-TaP.png}
        \tlabel{structure-Nb_TaP-100}
    }\\
    \subfloat[]{
        \ig{0.31}{structure/NbP-unit-cell.png}
        \tlabel{structure-Nb_TaP-unit-cell}
    }
    \subfloat[]{
        \ig{0.31}{structure/wsm-all-layers-unit-cells.png}
        \tlabel{structure-all-unit-cell}
    }
    \caption{
        Lattice structure of
        \sfA~ (100) plane of MgO rotated by 45 \degree for clarity, 
        \sfB~  (100) plane of Nb(Ta),  
        \sfC~  (100) plane of Nb(Ta)P.
        \sfD~ Nb(Ta)P unit cell
        \sfE~ Nb(Ta)P/Nb(Ta)/MgO unit cell,
    }
    \tlabel{structure-wsm}
\end{figure}

The choice of substrate which is lattice matched to Nb(Ta)P is a bit limited by the 
fact that we need insulating substrate for fabricating electrical and spintronic 
devices. MgO is one such substrate which can easily be treated to become atomically
flat and is lattice matched to [100] direction of NbP with [110] direction in the
substrate as shown in \tref{structure-wsm} and tabulated in \tref{wsm-lattice-table}

\section{Substrate Preparation}

Nb(Ta)P were grown in TAMARIND using MBE on MgO substrates, usually 5$\text(\times)$10 mm in 
size from the supplier Crystec. The substrate was prepared for growth as follows:

\begin{enumerate}
    \item Soak the substrate in methanol for 20 minutes.
    \item Rinse thoroughly in deionized (DI) water thoroughly and blow with dry nitrogen gas.
    \item Anneal the substrate in O2 atmosphere at \temperature{1050 - 1100} for 3 h.
\end{enumerate}

\begin{figure}
    \centering
    \subfloat[]{
        \ig{0.31}{afm/MgO-with-pits.png}
        \tlabel{afm-MgO-pits}
    }
    \subfloat[]{
        \ig{0.31}{afm/MgO-usable-sample.png}
        \tlabel{afm-MgO-usable}
    }\\
    \subfloat[]{
        \ig{0.31}{afm/Ls-MgO-with-pits.png}
        \tlabel{afm-MgO-pits-ls}
    }
    \subfloat[]{
        \ig{0.31}{afm/Ls-MgO-usable-sample.png}
        \tlabel{afm-MgO-usable-ls}
    }
    \caption{
        AFM of  an MgO substrate 
        \sfA~ with pits after cleaning, 
        \sfB~ typical usable sample.
        Step scan of the substarte  
        \sfC~ with pits after cleaning,
        \sfD~ typical usable sample.
    }
    \tlabel{afm-MgO}
\end{figure}

The process is not fully reproducible. Most of the times we would get atomically flat substrates 
but occasionally we would see pits and sometimes the terrace are not of atomic step height. 
The representative figure of both such scenarios have been shown in \tref{afm-MgO-pits} and
\tref{afm-MgO-usable}  with their line scans shown \tref{afm-MgO-pits-ls} and \tref{afm-MgO-usable-ls}

\section{Evaporation of materials}

Ta(Nb) is evaporated from a rod using electron-beam heating and P is evaporated using a 
GaP effusion cell at \temperature{850}in TAMARIND chamber. The residual P in the chamber sticks to 
the chamber in two major allotropic forms, \ptwo (red Phosphorous) and \pfour (white 
phosphorous). The red phosphorus is thermodynamically stable at room temperature but 
white phosphorous can spontaneously combust at room temperature and turn in lethal phosphine 
\phosphine gas. Therefore, it is absolutely necessary to keep the white phosphorus
concentration to as low value as possible. We observe that the ration of 
\ptwo : \pfour is around 1:100.

\section{Growth of Nb(Ta)P}


There is more than 10\% mismatch between the lattice parameters of Nb(Ta)P and MgO.
If we try to grow these compounds directly on the substrate, the film does not remain 
crystalline; we start getting islands of these films instead of single crystalline flat 
films.  It turns out that if we grow a buffer layer of Nb(Ta) before the growth of 
phosphide, we get high quality single crystalline films. This buffer layer is grown at 
\temperature{300}, at a rate of 3-5 nm/h and a pressure of \pressure{4}{-10}. The surface 
of the buffer layer is then exposed to a P2 flux (BEP: \pressure{1}{-8}) to achieve phosphorization.
Once the buffer layer has been phosphorized the surface is ready for growth of Nb(Ta)P. 
The substrate temperature is maintained between 300 to 400 ºC and the film is grown at a 
slow rate of < 4nm/h which is limited by the Nb (Ta) flux. The P flux is at least 20 times 
higher than the Nb flux and has BEP of \pressure{1}{-8}. The sample is cooled down at the rate of 
10 \degree C/min in P-atmosphere, to ensure that P-termination surface is homogeneous. 


\section{Structure of Weyl NbP and TaP}
\begin{figure}
    \centering
    \subfloat[]{
        \ig{0.31}{rheed/MgO_100.png}
        \tlabel{rheed-MgO-100}
    }
    \subfloat[]{
        \ig{0.31}{rheed/MgO_110.png}
        \tlabel{rheed-MgO-110}
    }\\
    \subfloat[]{
        \ig{0.31}{rheed/Nb_100.png}
        \tlabel{rheed-Nb-100}
    }
    \subfloat[]{
        \ig{0.31}{rheed/Nb-110.png}
        \tlabel{rheed-Nb-110}
    }\\
    \subfloat[]{
        \ig{0.31}{rheed/NbP_100_begin.png}
        \tlabel{rheed-NbP-100-begin}
    }
    \subfloat[]{
        \ig{0.31}{rheed/NbP_110_begin.png}
        \tlabel{rheed-NbP-110-begin}
    }\\
    \subfloat[]{
        \ig{0.31}{rheed/NbP_100_end.png}
        \tlabel{rheed-NbP-100}
    }
    \subfloat[]{
        \ig{0.31}{rheed/NbP_110_end.png}
        \tlabel{rheed-NbP-110}
    }
    \caption{
        MgO substrate before growth in
        \sfA~ \direction{100} 
        \sfB~ \direction{110}.
        Nb buffer layer in
        \sfC~ \direction{100} 
        \sfD~ \direction{110}.
        Nb buffer layer after posphorization in
        \sfE~ \direction{100} 
        \sfF~ \direction{110}.
        NbP thin film in
        \sfG~ \direction{100} 
        \sfG~ \direction{110}.
    }
    \tlabel{rheed-wsm}
\end{figure}

\begin{figure}
    \centering
    \subfloat[]{
        \ig{0.31}{xrd/NbP_theta2theta.png}
        \tlabel{xrd-NbP-theta}
    }
    \subfloat[]{
        \ig{0.31}{xrd/NbP_rsm.png}
        \tlabel{xrd-NbP-rsm}
    }
    \caption{
        \sfA~ Theta-2Theta Scan and 
        \sfB~ RSM scan of 
        NbP.
    }
    \tlabel{xps-wsm}
\end{figure}


The film quality is constantly monitored using in-situ reflection high energy electron diffraction 
(RHEED) with 15 kV energy and 1.5 mA filament current. We take an atomically flat terraced MgO 
substrate (\tref{rheed-MgO-100} and \tref{rheed-MgO-110}) and grow a Nb (Ta) buffer layer on it 
in order to minimize the lattice mismatch. These layers are rotated 45\degree\ with respect to 
each other i.e. \direction{100} direction of MgO points along the \plane{110} plane of Nb (Ta) 
as seen in \tref{rheed-Nb-100} and \tref{rheed-Nb-110}. Before starting the growth of Nb(Ta)P 
we phosphorize the surface in order to minimize the thickness of the buffer layer. The higher 
order streaks appear during this process and the RHEED pattern becomes a bit more diffusive as 
shown in \tref{rheed-NbP-100-begin} and \tref{rheed-NbP-110-begin}. It suggests that the grain 
size is not very large yet. Once the growth starts and grains start to coalesce the RHEED pattern 
becomes very streaky which suggests that the film starts to follow the terraces of the substrate. 
The periodic RHEED reflections are only visible in the high symmetry directions (45\degree\ 
periodicity), while there are no coherent patterns at intermediate angles, indicating that the 
NbP (TaP) films grow with a single-crystalline orientation without twinning/twisting of in-plane 
crystalline domains (\tref{rheed-NbP-100} and \tref{rheed-NbP-110}).

A standard θ-2θ scan taken on NbP films of various thicknesses (9-70 nm) shows only 
(004) and (008) NbP reflections (Fig. 2a), confirming an epitaxial, single crystalline 
oriented growth without secondary phases. In order to compare the quantification using 
both local and global methods, a reciprocal space mapping (RSM) scan has been performed 
on the (1,1,10) reflection of NbP (Fig. 2d), yielding an in-plane lattice parameter of 
$a=3.391 \angstrom$. The in-plane lattice parameters do not vary from as the thickness 
is varied from 15nm to 70nm according to the RSM measurements (see Sup. Fig S2), which 
means that the films do not change the in-plane strain state in the studied thickness range.
On the other hand, the out-of-plane lattice parameters extracted from the (004) peak positions
(inset of Fig. 2a) decrease only very slightly (11.50 Å to 11.46 Å) with increasing thickness.
The negligible thickness dependence of both in-plane and out of-plane parameters suggest that 
the films grow fully relaxed from the very early stage. Interestingly, the lattice parameters 
($a=3.39 \angstrom, c=11.48 \angstrom$) sizably differ from the bulk values 
($a=3.34 \angstrom, c=11.37 \angstrom$), which means that a bigger unit cell is stabilized 
during the layer-by-layer growth on the MgO(001)/Nb(001) surface.



\section{Morphology of TaP and NbP}

\begin{figure}
    \centering
    \subfloat[]{
        \ig{0.31}{afm/NbP_afm.png}
        \tlabel{afm-NbP}
    }
    \subfloat[]{
        \ig{0.31}{afm/TaP_afm.png}
        \tlabel{afm-TaP}
    }\\
    \subfloat[]{
        \ig{0.5}{afm/NbP_line_scan.png}
        \tlabel{afm-NbP-ls}
    }
    \caption{
        \sfA~ AFM of NbP and 
        \sfB~ AFM of TaP.
        \sfC~ Line Scan of NbP
    }
    \tlabel{afm-wsm}
\end{figure}

The topography and surface structure of NbP and TaP films is of paramount
importance for the observation of topological surface states (Fermi-arcs) 
and was thus investigated by scanning probe microscopy (Figure 4 and Supp. Fig S5).  
Large-scale atomic force microscopy images (Supp Fig. 5a) reveal a flat topography, 
yielding root mean square (RMS) roughness values of 0.43 nm. The grain size varies 
from 50 to 100 nm showing regions of grain coalescence, whereas the inter-grain steps 
correspond mostly to 1 unit cell height.  In order to investigate the topography inside 
and between the grains, in-situ scanning tunneling microscopy images have been acquired 
(Figure 4). Square-and rectangular shaped grains can be identified with two preferred 
orientations, along 45\degree\ and -45\degree\ on the image axis (i.e. along (100) direction), consistent 
with the four-fold in-plane crystal symmetry of NbP. A closer look to the grain topography 
reveals the presence of atomically flat terraces. The height of each step terrace amounts to 
2.8 \angstrom, corresponding to ¼ unit cell fractions (a single Nb-P monolayer), 
as depicted in Figure 2b (a structural model of the NbP unit cell along the growth 
direction (001) is drawn as guide to the eye). We rarely find slight deviations of 
exact unit cell fractions, which might arise if adjacent grains end in a different 
(Nb / P) atomic termination or due to the presence of inter-grain stacking faults. 
From the topography statistics, a predominantly single surface termination (either Nb or P)
scenario is likely to happen throughout the film surface. Figure 4c and 4d show a 
zoomed-in topography image comparing NbP and TaP film surfaces, the latter having a 
smaller terrace width.


\section{Stoichiometry of Wel Semimetals}

\begin{figure}
    \centering
    \subfloat[]{
        \ig{0.31}{xps/Nb_3d.png}
        \tlabel{xps-Nb-3d}
    }
    \subfloat[]{
        \ig{0.31}{xps/NbP-P-2p.png}
        \tlabel{xps-NbP-2p}
    }\\
    \subfloat[]{
        \ig{0.31}{xps/Ta-3d.png}
        \tlabel{xps-Ta-3d}
    }
    \subfloat[]{
        \ig{0.31}{xps/TaP-P-2p.png}
        \tlabel{xps-TaP-2p}
    }
    \caption{
        \sfA~ Nb 3d peak and
        \sfB~ P 2p peak in 
        NbP. 
        \sfC~ Ta 3d peak and
        \sfD~ P 2p peak in 
        TaP.
    }
    \tlabel{xps-wsm}
\end{figure}

\begin{figure}
    \centering
    \subfloat[]{
        \ig{0.31}{rbs/NbP_rbs.png}
        \tlabel{rbs-NbP}
    }
    \subfloat[]{
        \ig{0.31}{rbs/TaP_rbs.png}
        \tlabel{rbs-TaP}
    }
    \caption{
        RBS spectra of 
        \sfA~ NbP
        \sfB~ TaP.
    }
    \tlabel{rbs-wsm}
\end{figure}

\begin{figure}
    \centering
    \subfloat[]{
        \ig{0.31}{edx/NbP_edx.png}
        \tlabel{edx-NbP}
    }\\
    \subfloat[]{
        \ig{0.6}{edx/TaP_edx.png}
        \tlabel{edx-TaP}
    }
    \caption{
        EDX Spectra of
        \sfA~ NbP
        \sfB~ TaP.
    }
    \tlabel{edx-wsm}
\end{figure}

The chemical composition and valence states of the NbP (TaP) thin films have
 been studied by in-situ X-ray photoemission spectroscopy (XPS) and ex-situ 
 Rutherford backscattering spectroscopy (RBS).  The Nb 3d 5/2 and P 2p core 
 level peaks in XPS spectra is shown in \tref{xps-wsm}. The figure shows that Nb 
 and P peaks shift in opposite directions compared to the neutral Nb0 and P0 
 valence state. This is consistent with the expected chemical shifts due to 
 electron transfer in the NbP compound (NbIII and PV valence). The stoichiometry 
 of the films has been determined using equation (37), yielding a slightly P-rich 
 (Nb0.49 P0.51) composition. In addition to XPS, which is a surface sensitive technique,
  the in-depth composition of the films has been studied by Rutherford backscattering 
  spectroscopy (RBS). The best fitting to the spectra \tref{rbs-wsm}, yields a 
  47.6\% to 52.4\% (Nb:P) composition, indicating that the P-rich composition 
  is distributed homogeneously across the full NbP layer.  Similar results with regard 
  to bonding-related core-level shifts and P-rich composition have been found in TaP 
  layers which  are show in the same figure. A real-space visualization of the layer 
  homogeneity is further shown by energy-dispersive X-ray spectroscopy along a NbP and 
  TaP cross section is shown in \tref{edx-wsm}.

The fine details of the structure are shown in Figure 2c, where a high-resolution 
high-angle annular dark field (HAADF) STEM image of the NbP in the [110] direction 
displays highly-ordered in-and out-of-plane lattice planes, with an excellent matching
to the NbP structural model.  The lattice planes are particularly visible by the high 
atomic contrast of the Nb atoms, while a faint contrast corresponding to the light P 
atoms can be distinguished in the zoomed-in image (Figure 2c, right panel) at the 
expected atomic positions. Taking advantage of the excellent structural order, a line 
intensity profile of the atomic rows has been taken to calculate the average in-plane 
lattice parameter of the NbP film (Supp. Figure S1). For the [110] direction, a Nb-Nb 
atom distance of $2.40 \angstrom$ has been inferred, which results in an in-plane lattice parameter 
$a=3.394 \angstrom$.  

\section{Reciprocal space characterization of NbP}

\begin{figure}
    \centering
    \subfloat[]{
        \ig{0.4}{arpes/fermi-surface.png}
        \tlabel{edx-TaP}
    }\\
    \subfloat[]{
        \ig{0.7}{arpes/cacluated-fermi-surface.png}
        \tlabel{arpes-NbP-calc-fermi}
    }
    \caption{
        \sfA~ Calculated fermi surface of NbP.
        \sfB~ Measured fermi surface of NbP
    }
    \tlabel{fermi-wsm}
\end{figure}

\begin{figure}
    \centering
    \ig{0.9}{arpes/NbP_tem_weyl.png}
    \tlabel{arpes-tem-NbP}

    
    \caption{
       TEM and ARPES of NbP
    }
    \tlabel{arpes-tem-wsm}
\end{figure}

Having assessed the properties of the film surface, momentum-resolved 
photoemission spectra have been taken using an in-house designed and built 
momentum microscope45 with a He-I light source. Figure 5 summarizes the
overall electronic structure of a 15 nm-thick NbP thin film measured at 
100K, including Fermi-surface topology and band dispersion along relevant 
(Weyl point) cuts, together with ab-initio calculations using the experimentally 
obtained unit cell parameters. At the Fermi-energy, four electronic pockets with 
an elliptic shape directed towards the X and Y symmetry points can be identified, 
as shown in Fig. 5a.  A detailed comparison with the calculation of the 
termination-dependent surface states (Figures 5b and 5c) reveal that the elliptic 
and cross-like band features along Γ-X and Γ-Y are characteristic of the P-terminated 
NbP surface states (the features for a Nb-termination are radically different). 
Interestingly, the size and shape of the measured elliptical (also called spoon-like) 
features match with the calculations and the bulk crystal data in Refs.10,11,46 only 
when an energy shift of $\Delta E = 0.2 \text{eV}$. is considered (highlighted 
in a green box, Figure 5b), suggesting an effective hole doping in the as-grown 
thin films.  Figure 5d shows the energy dispersion cut along $A \rightarrow A'$, 
which is expected to cross the location of one pair of Weyl-points in NbP 
($k_x=0.54 \angstrom^{-1}, E_b = -0.026 eV$) and thus used to visualize the surface 
Fermi arcs. A clear linear band dispersion is observed, in agreement with previous 
photoemission results of cleaved monopnictide bulk crystals.6-11  This dispersion 
originates from the Fermi-arcs, but their kx-ky contour at the Weyl points cannot 
be mapped in our NbP films due to the EF shift (-0.2 eV) with respect to the intrinsic 
Fermi-level, and secondly, due the intrinsically short separation of the Weyl points 
in momentum space ($\Delta k < 0.05 \angstrom-1$), a detection challenge even for high-resolution 
synchrotron ARPES. It is noteworthy that only these topological surface states (elliptic 
shape, $A \rightarrow A'$ cut) are observed in our films, whereas the bowtie features
centered at the X and Y points) are completely absent.  This merits further investigation
and will be discussed elsewhere. On the other hand, the origin of the effective hole-doping 
in the as-grown thin films can be attributed to residual acceptors (Nb-vacancies) arising 
from the MBE growth process, in agreement with the slightly P-rich composition of the films 
inferred by XPS and RBS.  The strained lattice parameters of the phosphide thin films 
will also have an effect on the Fermi-level position. Although we estimate that the effect 
of the lattice parameters (~1\% tensile strain) is rather small, further studies are needed 
to disentangle the contribution from strain and vacancy acceptors on EF, in order to get a 
full understanding of Fermi-level engineering of Weyl semimetals achieved by epitaxial design.
