\addchapter{Spin to Charge Conversion}
\tlabel{spin-tocharge-conversion}
There are various methods which can be used to estimate the spin-to-charge conversion
efficiency of a material. I have used STFMR and second harmonic Hall analysis to do so.
The normal metal (NM) - ferromagnet (FM) heterstructure was lithographically patterned
into devices as shown in \tref{stfmr-pattern}. 

\begin{figure}
    \centering
    \ig{0.6}{stfmr.png}
    \tlabel{fig:stfmr-pattern}
    \caption{The STFMR design patterns}
\end{figure}

\begin{figure}
   
    \centering
    \subfloat[]{
        \ig{0.45}{nbse2_stfmr_f_dep.pdf}
        \tlabel{stfmr-nbse2-f}
    }
    \subfloat[]{
        \ig{0.45}{NbSe2_kittel.png}
        \tlabel{stfmr-nbse2-kittel}
    } \\
    \subfloat[]{
        \ig{0.45}{NbSe2_deltaH.png}
        \tlabel{stfmr-nbse2-kittel}
    }
    \subfloat[]{
        \ig{0.45}{NbSe2_stfmr_delta_delta.png}
        \tlabel{stfmr-nbse2-delta-delta}
    }
    \caption{ 
        \sfA~ STMFR mixed voltage of \nbse\ at various frequencies.
        \sfB~ Kittel fitting for determination of effective magnetization of Py in \nbse/
        /Py bilayer.
        \sfC~  Linewidth vs frequency of \nbse\ in \nbse/
        /Py bilayer.
        \sfD~ Linewidth boroadening vs current at positive and negative field during
        STFMR measurement.
    }
    \tlabel{fig:stmfr-nbse2}
\end{figure}

\section{Second harmonic Hall of \nbse}
The seond harmonic signal is fitted with according to the equations described
in \tref{spintronics-and-quantum-materials}. The magnetization is assumed to
be uniform while the transverse voltage is calculated using
standard expressions for the AHE, PHE, and ANE. The
simulations are repeated for positive and negative dc currents
forwhich the half of the difference and the average of these two
signals correspond to the equilibrium (current independent)
and current induced signals, respectively. This is equivalent
to Fourier-transformed first and second harmonic signals in
an ac current injection measurement. Note that, relative to
the simulations and depending on the system under study,
the direction and amplitude of the torques and ANE can
change sign in the experiment.



\section{STFMR of \nbse}

The gilbert damping of the permalloy films was found to be around 0.0085 $\pm$ 0.0003 
for all the films. 

% \section{Second Harmonic of Weyl semimetals}

% \section{Comparison of different materials}

