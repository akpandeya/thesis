\addchapter{Experimental Set Up}

\section{PAPAYA}
\label{papaya}


PAPAYA is growth-cum-analysis chamber which allows users to grow samples using
MBE and characterize them using various in-situ techniques. It has 3 MBE 
chambers, an XPS chamber and a low temperature STM-cum-Q plus AFM. The system 
have following features:

\begin{enumerate}
    \item Selenide MBE : We can grow various Metal Selenide in this chamber. 
    It's one port has Se and has 4 ports for other metals like Nb, Pd, V etc. 
    The substrate can rotate up to \temperature{320} and heated up to 
    \temperature{900} The chamber has RHEED for characterizing the thin 
    film while growing it.

    \item Telluride MBE: Various Tellurides can be grown in this chamber. 
    It has only 3 ports and has mateials like SnTe in it. It has not been 
    used in this work and hence, will not be described in detail here.

    \item Metal MBE: Despite its name we can grow both metals and non-metals 
    in it. It has 5 ports and Al, MgO, Py (Ni80Fe20), Ni and Cu  can be grown 
    in this chamber. The sample can rotated for \temperature{360} but can 
    be heated and hence, everything is grown at room temperature.

    \item In-Situ X-ray photoelectron spectra (XPS) system: XPS is a surface 
    sensitive techinique which can be used to characterize the elemental 
    composition of a material and chemical states of these elements. It is 
    a standard characterization tool based of photoelectric effect. When 
    X-Ray with energy ($\photonEnergy$) with falls on a material with electron 
    withbinding energy ($\bindingEnergy$) and surface work function ($\phi$), 
    electrons with energy ($\kineticEnergy$) are emitted.
    \begin{equation}
        \tlabel{binding}
        \bindingEnergy = \photonEnergy - \left( \kineticEnergy + \phi \right) 
    \end{equation}

    The kinetic energy of the emitted electrons is measured using a 
    hemispherical analyser and using the \tref{binding} we can estimate the 
    binding energy of the electrons. The XPS spectrum is number of detected 
    electrons as a function of binding (or kinetic energy). A peak is observed 
    corresponding to the binding energies of the electrons present in an 
    element. Every element has electrons at fixed binding energies and thus, 
    a set of peaks can serve as a fingerprint of the element. The position of 
    the peaks may increase or decrease by a little bit based on its chemical 
    state. The relative concentration of various elements can be estimated by 
    taking the ratio of area under the curve of these peaks after taking 
    relative sensitivity factor into account and subtracting background 
    properly.

    \begin{equation}
        \tlabel{cps-conc}
        \frac{C_1}{C_2} = \frac{A_1 \times r_2}{A_2 \times r_1}
    \end{equation}
    An omicron machine with Al and Mg sources is attached to PAPAYA. 
    The measurements can be done only at room temperature in this system.
    
    \item In-Situ Scanning Tunneling Microscope (STM): The STM experiments 
    were performed on an Omicron VT-STM-XT system operated at room temperature 
    with a base pressure of \pressure{2}{-11}. The mechanically sharpened 
    Pt/Ir tips were treated and checked on Au(111) surface before measurements, 
    and the topography images were acquired at room-temperature.
\end{enumerate}

\begin{figure}
    \ig{0.5}{papaya.png}
    \tlabel{papaya}
    \caption{
        PAPAYA System
    }
\end{figure}


\section{TAMARIND - Topological materials engineering by epitaxial design}
It can be used to phosphides and arsenides. It has 7 ports in which Nb, Ta, 
Py, MgO, GaP (as source of P) and GaAs (as source of As) is attached. 
The substrate can be rotated up to 320 \degree\ and heated up to \temperature{1200} using 
resistive heating. To achieve higher substrate temperature there is a 
possibility to use e-beam heating. The substrate can be cooled using liquid 
N2 to up to \temperature{-30}. There is a RHEED to monitor the growth of the film. There is a 
residual gas analyser and a mass spectrometer to monitor the partial pressure of various 
gaseous species in the chamber. 

\begin{figure}
    \ig{0.5}{tamarind.png}
    \tlabel{tamarind}
    \caption{
        TAMRAIND System
    }
\end{figure}
\section{STFMR Set-Up}

\begin{figure}
    \ig{0.5}{stfmr-schematic.png}
    \tlabel{stfmr}
    \caption{
        STFMR Schematic
    }
\end{figure}

\section{Second Harmonic Hall Set Up}
