\addchapter{Introduction}
\label{introduction}

\section{Moore’s Law – Room at the bottom is shrinking}
The number of transistors per unit area in electronic circuit keeps doubling almost every two years. 
It was first observed by Gordon Moore in his 1965 paper. In many ways Moore’s law has been surpassed 
so far. Aggressive scaling of electronic devices which has followed Moore’s law has made “plenty of room” 
for storage and devices as Feynman had once envisaged in his famous lecture. But pace of growth is plateauing 
and the there is a limit to where CMOS can take us. 

Memory devices has similarly experienced a lot of growth dues to scaling but we are reaching the limits of what
can be achieved. The main form in which data is stored is magnetic bits and these bits are read and written using
magnetic read/write head. 

\section{Spintronics for scaling and beyond}
The key to achieving this is making sure that any internal status of
computation is stored before power is turned off, without consuming
power. Non-volatile RAM is therefore a critical component. It
offers an infinite number of fast write and read operations as well
as non-volatility. Furthermore, this memory is free of soft-errors
caused by radiation. Spin-transfer torque RAM

\section{Quantum materials for spintronics}
Nominally highly spin-polarized materials, as discussed
in the previous sections, could provide both effective
spin injection into nonmagnetic materials and
large magnetoresistance effects, important for nonvolatile
applications. Examples include half-metallic oxides
such as CrO2, Fe3O4 , CMR materials, and double per
Most of the currently studied ferromagnetic semiconductors
are p-doped with holes as spin-polarized carriers,
which typically leads to lower mobilities and
shorter spin relaxation times than in n-doped materials.
It is possible to use selective doping to substantially increase
Tc , as compared to the uniformly doped bulk
ferromagnetic semiconductors

\section{Dichalcogenides}

Graphene is not the only prominent example of two-dimensional (2D) materials. Due to their
interesting combination of high mechanical strength and optical transparency, direct bandgap
and atomic scale thickness transition-metal dichalcogenides (TMDCs) are an example of
other materials that are now vying for the attention of the materials research community. In this
article, the current state of quantum-theoretical calculations of the electronic and mechanical
properties of semiconducting TMDC materials are presented. In particular, the intriguing
interplay between external parameters (electric fi eld, strain) and band structure, as well as
the basic properties of heterostructures formed by vertical stacking of different 2D TMDCs
are reviewed. Electrical measurements of MoS 2 , WS 2 , and WSe 2 and their heterostructures,
starting from simple fi eld-effect transistors to more demanding logic circuits, high-frequency
transistors, and memory devices, are also presented.

\section{Weyl semi-metals}
Weyl semimetals are semimetals
or metals whose quasiparticle
excitation is the Weyl fermion, a
particle that played a crucial role in quantum
field theory but has not been observed as a
fundamental particle in vacuum1-24. Weyl
fermions have definite chiralities, either
left-handed or right-handed. In a Weyl
semimetal, the chirality can be understood as
a topologically protected chiral charge. Weyl
nodes of opposite chirality are separated in
momentum space and are connected only
through the crystal boundary by an exotic
non-closed surface state, the Fermi arcs.
Remarkably, Weyl fermions are robust while
carrying currents, giving rise to exceptionally
high mobilities. Their spins are locked to
their momentum directions, owing to their
character of momentum-space magnetic
monopole configuration. Because of the
chiral anomaly, the presence of parallel
electric and magnetic fields can break the
apparent conservation of the chiral charge,
making a Weyl metal, unlike ordinary nonmagnetic
metals, more conductive with
an increasing magnetic field. These new
topological phenomena beyond topological
insulators make new physics accessible and
suggest potential applications, despite the
early stage of the research
\section{Outline}
The theis is organized as follows:

\begin{itemize}
    \item Chapter 2 discusses the theoretical background required to 
    understand spin to charge conversion and the methods required to measure them
    \item Chapter 3 discusses the experimental set up used to perform the experiments
    \item Chapter 3 discusses growth and characterization of \nbse
    \item Chapter 4 discusses the growth and characterization of weyl semimetals
    \item Chapter 5 discusses the electrical properties of dichalcogenides and WSMs
    \item Chapter 6 discusses the spin-tocharge conversion in WSMs and dichalcogenides
\end{itemize}
