\addchapter{Introduction}
\label{introduction}

\section{Moore’s Law – Room at the bottom is shrinking}
The number of transistors per unit area in electronic circuit keeps doubling almost every two years. 
It was first observed by Gordon Moore in his 1965 paper. In many ways Moore’s law has been surpassed 
so far. Aggressive scaling of electronic devices which has followed Moore’s law has made “plenty of room” 
for storage and devices as Feynman had once envisaged in his famous lecture. But pace of growth is plateauing 
and the there is a limit to where CMOS can take us. 

Memory devices has similarly experienced a lot of growth dues to scaling but we are reaching the limits of what
can be achieved. The main form in which data is stored is magnetic bits and these bits are read and written using
magnetic read/write head. 

\section{Spintronics for scaling and beyond}


\section{Quantum materials for spintronics}

\section{Dichalcogenides}

\section{Weyl semi-metals}

\section{Outline}
