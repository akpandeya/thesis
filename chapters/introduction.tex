\addchapter{Motivation and Background}
\label{Introduction}


\section{Emergent properties in quantum materials}
Emergence is appearance of a new property in a collection of items where none of 
the individual item has that property. \cite{Anderson1972}.
The collective behaviour of electrons in a solid is different than behavior of a single electron. 
Such properties of electrons are also known as emergent properties of the material. These often arise due to interaction of different 
electronic sates in a solid. It is important to understand the how charge, spin and and orbital degree
of freedom interact in quantum materials to give rise to emergent properties like topology,
superconductivity etc not just for advancing the fundamental science but also for making the
next generation of devices beyond CMOS \cite{TokuaraEmergent}.

The number of transistors per unit area in electronic circuit keeps doubling almost every two years. 
It was first observed by Gordon Moore in his 1965 paper. In many ways Moore's law has been surpassed 
so far. Aggressive scaling of electronic devices which has followed Moore's law has made \emph{plenty of room}
for storage and devices as Feynman had once envisaged in his famous lecture \cite{feynman1960engineering}. But pace of growth is plateauing 
and the there is a limit to where complementary cetal oxide semiconductor (CMOS) can take us. 

The main form in which data is stored is magnetic bits and these bits are read and written using
magnetic read/write head. Memory devices have similarly experienced a lot of growth dues to 
scaling but we are reaching the limits of what can be achieved with traditional materials \cite{Bader2010}.


\section{Spintronics for scaling and beyond}
The Random Access Memory (RAM) encodes the information in a capcitor using charge 
of the electron. It means that we need to recharge the capacitors very often in order 
to persist the information. As the device size shrinks, the  thickness of insulating layer
needs to be decreased in order to get to the same voltage levvel. This means that thickness of 
the insulating layer is decreased and hence leakage current increases which leads to wasted energy.
Thus by decreasing the size of RAM leads to higher refresh rate for capacitor and more wasted energy
on recharging capacitor \cite{Jacob2008}.

This problem can be solved by switching from the conventional RAM to Magnetic Random Access Memory (MRAM). 
It consists of a three layers - an insulating oxide sandwiched between feromagnets. 
When magnetizations of the two ferromagnetic layers are parallel to each other the the 
resistance of the stack is low and current can be easily passed through it. On the other 
hand when the magnetizations are anti-parallel the resistance of the stack is high and no 
current can be passed through it. Thus, we can store 0 and 1 state in the 
this device \cite{Inomata2001}.

One of the ferromagnets - called free layer - can be easily flipped from one 
magnetization to another by passing electrons one kind of spin.This mechanism 
of flipping the magnetization of a ferromagnet by using spin of the electrons
is called spin transfer torque(STT). The other ferromagnet - called fixed layer - has higher
coercivity and cannot easily switched. This is usually achieved by making the fixed layer 
thicker than the free layer. By changing the magnetization of the free layer 
using STT we can write information into this MRAM and hence is also called
STT-MRAM \cite{Hirota2002}.

Unfortunately, the STT mechanism has inherent drawback that the current must pass through
the thin oxide layer. This produces a lot of heat and may damage the information stored in the
MRAM. Another mechanism called spin-orbit torque (SOT) was proposed to overcome this limitation.
The spin current is generated using other means and the reading and writing path of the 
MRAM is spearted. This means that we no longer have to pass current through the oxide
layer to write information in the SOT-MRAM \cite{spintronic-sarma}.

\section{Quantum materials for spintronics}
We need materials which can generate spin current to write information in
SOT-MRAMs. A number of rare earth metals like Pt, W etc.  have been found to be very efficient
in converting charge current into spin current. We need to explore more such materials
(i) to understand the mechanism of charge to spin conversion and (ii) explore materials
which can have possibly higher spin-to-charge conversion efffciency. The two classes of
materials which can be further explored are (i) Transition Metal Dichalcogenides (TMDs) and 
(ii) Weyl semimetals \cite{TokuaraEmergent}.

\section{Dichalcogenides}

Transition Metal Dichalogenides  or just dichalcogenides are materials of the formula
\ce{MX2}, where where M is a transition metal element and X is a chalcogen atom. Transition
metals belong to group IV - VI of periodic table while chalcogen atom is either S, Se
or Te. Like graphene these materials can be easily exfoliated and studied and therefore,
wide range of experiments have been performed on them. But the experiments to study 
spin-orbit torque require us to make clean heterostructures. Exfoliation
is not best suited method to make such heterstructures. A clean method like molecular beam
epitaxy would be better-suited for such experiments \cite{Manzeli}.

\section{Weyl semimetals}

Weyl fermions are massless fermions. Weyl semimetals (WSMs) are metals/semimetals 
in which electrons whose quasiparticle excitation is a Weyl fermion i.e. they behave 
like a Weyl fermion to external applied fields i.e. the effective mass of the electrons
is 0. Since effective mass is given by second derivative of bandstructure, which means that
the dispersion relation is linear. The key characteristic is that electrons are filled till
the nodes and they have well-defined chiralities viz. left-handed or right-handed. The 
spin and momentum of the electrons are locked to their momentum direction. These materials
may have a high spin-to-charge conversion efffciency \cite{Bansil2016}.

\section{Outline}
The ability to grow high quality interfaces to achieve spin transfer torque in the materials 
is very important for continued scaling of memory and computation devices. The main motivation 
behind this work is to demonstrate growth of high quality thin films which can be easily doped
and  strained to fabricate functional devices and heterostructures. In this work, 
I have demonstrated growth of weyl semimetals for the first time using molecular beam epitaxy, 
grown heterstructures and measured the spin Hall angle of these quantum materials.
The thesis is organized as follows:

\begin{itemize}
    \item Chapter 2 discusses the theoretical background required to 
    understand spin to charge conversion and the methods required to measure them.
    \item Chapter 3 discusses the experimental set up used to perform the experiments. 
    State-of-the-art molecular beam epitaxy (MBE) cluster called Multi-purpose chamber 
    for production and analysis of nano-systems (PAPAYA) was assembled as part of my PhD.
    has been described. The growth chamber required to grow phosphides requires a lot of 
    care because of the high reactivity and hazardous nature of various phosphides.
    \item Chapter 3 discusses how the \nbse\ has been grown and various characterization
    performed to assess the crystal structure, stoichiometry and purity of the films.
    \item Chapter 3 discusses how the weyl semimetals (WSMs) have been grown and various characterization
    performed to assess the crystal structure, stoichiometry and purity of the films.
    \item Chapter 5 discusses the measurements performed on the \nbse\ and WSMs using ppms.
    The resitance vs temperature behavior of the lithographically prepared devices
    is measured. 
    \item Chapter 6 discusses the spin-to-charge conversion in WSMs and dichalcogenides
    in thin film-permalloy heterstruture using STFMR.

\end{itemize}
