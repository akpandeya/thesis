\addchapter{Spintronics and quantum materials}
\tlabel{ch:spintronics-and-quantum-materials}

\section{Spin current - two current model}
At a very basic level, spin current is net transfer of spins from one point to another. 
Spin current is actually defined as a tensor $q_{ij}$, where $i$ denotes the direction of 
flow of spin current and j denotes the which component of spin is flowing. The indices 
$i$ and $j$ are the directions in the 3D space. We can, in principle, have a net spin current
without any flow of charge current, if equal amount of up and down electron are moving in 
opposite direction. That is why these are often touted as dissipationless. But in reality, 
one always needs some amount of charge current to create the spin current in the first place 
which should be taken into consideration for making a fair comparison \cite{spintronic-sarma}.

There can be a net spin current if there is an imbalance between the up spin and down spin 
flowing in the opposite direction to each other. The electrical spin polarization
$P_j$ in the direction $j$  can be defined as follows:
\begin{equation}
    \centering
    \tlabel{spin-pol-def}
    P_j=\frac{n^{\uparrow}-n^{\downarrow}}{n^{\uparrow}+n^{\downarrow}}
\end{equation}

\section{How to create spin current?}

The spin current can be created using many methods. One of the simplest methods is to take 
a ferromagnet and pass current through it and inject the current coming out of the 
ferromagnet into wherever we might need (). But this is a fairly inefficient method. 
There are other methods like spin pumping, spin Hall effect and inverse spin galvanic 
effect (Rashba-Edelstein effect) etc. These methods will be described in the following chapter.

\subsection{Spin Hall effect (SHE)}

\begin{figure}
    \ig{0.5}{SHE.pdf}
    \tlabel{she}
    \caption{Schematic of SHE (blue dots represent electron with up spin and
    red dots represent the electrons with down spin)}
\end{figure}
The generation of transverse spin current on passing electrical current through a metal/semimetal is called
\textit{Spin Hall Effect}(SHE). The material must have spin-orbit coupling to show SHE. The current 
density ($\vb{J}$) and spin current density ($J_{ij}$) are given by:

\begin{equation}
    \tlabel{current-density}
    \vb{J}/e = \mu n e\vb{E} + D \vb{\nabla} n + \beta \vb{E \times P} + 
    \delta \vb{\nabla \times P}
\end{equation}

\begin{equation}
    \tlabel{spin-density}
    J_{ij} = - \mu E_{i} P_{j} + -D \frac{\partial P_{j}}{\partial x_{j}} + 
    \varepsilon_{ijk} \left(\beta n E_{k} + \delta \frac{\partial n}{\partial x_{k}}\right)
\end{equation}

\begin{equation}
    \tlabel{continuity}
    \partialderivative{P_{j}}{t} + \partialderivative{J_{ij}}{x_{i}} + 
    \partialderivative{P_{j}}{\tau_{s}} = 0
\end{equation}

Here, $J_{ij}$ means spin current with electrons polarized in $j$ direction and the spin current is
is flowing in $i$ direction, $\mu$ is the mobility of electrons, $n$ is the density of electrons, 
$\vb{P}$ is the vector of spin polarization, $i, j, k \in (x, y, z)$ and $\varepsilon$ is the 
unit antisymmetric matrix. \Tref{current-density} and \tref{spin-density} along with the continuity
equation (\tref{continuity}) give the complete pheomenological expressions for the current density and 
spin density.

The spin density relates to electric field as follows:
\begin{equation}
    J_{ij} = \sigma{ij} E_{i}
\end{equation}

where $\sigma{ij}$ is called spin hall conductivity. There are three microscopic mechanisms 
proposed for SHE and each of them contribute to the spin hall conductivity (SHC) \cite{Hirsch1999, Dyakonov2009, Sinova2015}.

\begin{enumerate}
    \item Intrinsic contribution: This contribution to the SHC depends only on the bandstructure
    of the material. For example, a Rashba like system may lead to spin polarization on the surface 
    when electric current is passed through it. It is therefore well defined for a given material
    and can be calculated \textit{ab-initio}.
    \item Skew scattering contribution: It arises due to spin-orbit coupling of the electrons and imurities.
    It is defined as the contribution to the SHE which is proportional to the lifetime of a Bloch state. 
    It usually dominates in the  nearly perfect crystals.
    \item Side jump contribution: The elctron od opposite spins are deflected to opposite directions
    when approaching and leaving the impurity due to the electric fields of the impurity. Its 
    conntribution is defined as difference of total spin hall conductivity from the sum of intrinsic 
    and skew mechanism part.
\end{enumerate}

We usually make a heterostructure of a magnetic material and spin-orbit torque material. 
How does this spin current affect the magnetization of the magnetic material? Landau-Lifshitz-Gilbert 
equation is disucssed in the following section in order to understand this.

% \section{How to measure spin current?}

% Spin current cannot, of course, be measured directly, hence we need convert to spin current into 
% charge current. The inverse phenomenon to the SHE is called Inverse Spin Hall Effect (ISHE).

\section{Landau-Lifshitz-Gilbert equation}
\tlabel{llg-section}
The magnetization of a magnetic material experiences a torque perpendicular to the direction of 
applied field and the direction of magnetization. The magnetization can keep precessing around 
the applied field forever in absence of any damping . But in real materials the radius of 
precession keeps decreasing and eventually the magnetization points towards the applied 
magnetic field.  The magnetization ($\mbold$) changes with effective field ($\hbold$) acoording
to following equation which is known as Landau-Lifshitz-Gilbert equation \cite{Hickey2009}:

\begin{equation}
    \tlabel{llg}
    \mdot = -\gamma(\mbold \times \hbold)+ \frac{\alpha}{M} \left(\mbold \times \mdot)\right)
\end{equation}
where $\alpha$ is Gilbert damping parameter. 
Slonczewski suggested a correction to term to the above equation when spin current (${J_{s}}$) in 
$\hat{\sigma}$ is the direction of injected spin current pumped into the material. 

\begin{eqnarray} \nonumber
    \tlabel{slonczewski}
    \mdot &=& -\gamma(\mbold \times \hbold) + \frac{\alpha}{M} \left(\mbold \times \mdot)\right) +
             \gamma \frac{\hbar}{2e \mu_{0} M_{s} t} \times J_{s} (\mbold \times \hat{\sigma} \times \vb{M})\\
            && - \left( \gamma \vb{M} \times \vb{H_{ref}} \right)
\end{eqnarray}

When a current is passed through a heterstructure of magnetic material and spin Hall material,
the spin Hall material produces spin current which diffuses into the magnetic layer. The change
in behavior of magnetic material may tell us how much spin current was produced. And, thus, we can
estimate the spin to charge conversion efficiency. Two such techniques are discussed below: 
(i) Spin Torques Ferromagnetic Resonance (STFMR) \cite{Resonance2019} and (ii) Second Harmonic Hall Measurement \cite{Harder2016, Avci2014f}.


\section{Spin torque ferrmomagnetic resonance - STFMR}

\begin{figure}
    \ig{0.5}{stfmr-schematic.png}
    \tlabel{stfmr}
    \caption{
        STFMR Schematic
    }
\end{figure}

A typical schematic of STFMR is shown in \tref{stfmr}. Magnetic field is applied in an in-plane
direction such that saturation magnetisation is reached. This system has a "natural frequency" 
of precession which depends on the applied magnetic field and the relative permeability of the magnetic
material. When an rf current of the same frequency is applied the system goes into resonance.
When we apply a dc current in addition to the rf current the resonance properties of the system is changed.
In particular, we can change the linewidth of the resonance by changing the amount of dc current.
In this resonance experiment it is impractical to sweep the applied frequency because the 
resistance of system also changes with the frequency. That is why sweep the natural frequency by
sweeping the applied magnetic field. Unless otherwise stated, in our experiments we keep the applied
field at 45 \degree or 135 \degree to the current diection \cite{Resonance2019}.

When a dc magnetic field $H_{ext}$ is sweeped in the STFMR configuration, the mixing 
voltage [10], can be expressed as in \tref{llg}:
\begin{equation}
    \tlabel{vmix}
    V_{mix} = V_{0} + V_{sym} \lorsym + V_{as} \lorasym
\end{equation}

We fit this equation to extract fitting parameters \vo, \hres, \vsym, \vas, \hres, and $\Delta$. 
When there is no DC current passing through the metal layer, the resonance field can be used to 
fit effective magnetization using Kittel formula [11] using \tref{vmix}.

\begin{equation}
    \tlabel{kittel}
    f_{res} = \frac{\gamma}{4\pi} \sqrt{(H_{res} + H_{0})(H_{res} + H_{0} + M_{eff})}
\end{equation}

The Gilbert damping can be extracted from the linewidth of the resonance peak [11] using \tref{gilbert}.

\begin{equation}
    \tlabel{gilbert}
    \Delta H = \Delta H_{res} + \frac{2\pi f}{\lambda}\alpha
\end{equation}

When a dc current is passed through the normal metal it pumps spin current into the ferromagnetic 
layer. In presence of a spin current the equation (4) is modified as [12], [13]:

\begin{equation}
    \Delta H = \Delta H_{res} + \frac{2\pi f}{\lambda} \left( 
        \alpha + \frac{\sin{\phi}}{(H_{ext} + 0.5M_{eff}) \mu_{0} M_{s} t}  
        \frac{\hbar}{2e}J_{s}
    \right)
\end{equation}

The change in linewidth ($\delta \Delta H$) for two different values of spin current 
($\Delta J_{s} = J_{s1} - J_{s2}$) will be:

\begin{equation}
    \tlabel{deltaDeltaH}
    \delta \Delta H = \frac{2\pi f}{\lambda} \left( 
        \alpha + \frac{\sin{\phi}}{(H_{ext} + 0.5M_{eff}) \mu_{0} M_{s} t} 
        \frac{\hbar}{2e}J_{s}
        \right)
\end{equation}

The \tref{deltaDeltaH} can be rearranged to get an expression of $\Delta J_{s}$:

\begin{equation}
    \tlabel{jseq}
    \Delta J_{s} = \frac{\delta \Delta H}{
        \frac{2\pi f}{\lambda} \left( 
        \alpha + \frac{\sin{\phi}}{(H_{ext} + 0.5M_{eff}) \mu_{0} M_{s} t} 
        \frac{\hbar}{2e}J_{s}
        \right)
    }
\end{equation}

We can assume a simple model of parallel connection between the ferromagnetic layer and 
the normal metal layer to find out the current distribution within the bilayer.


\begin{figure}
   
    \centering
    
    \ig{0.3}{parallel_fm_nm.png}
    \tlabel{parallel_fm_nm}

    \caption{Parallel connection between the ferromagnetic and normal metal layer}
    
\end{figure}

The current density through the metal layer is calculated as follows:
\begin{equation}
        J_{NM} = \frac{I_{NM}}{A_{c}} = \frac{V_{DC}/R_{NM}}{A_{C}} 
        = I_{DC} \frac{R_{FM} R_{NM}}{R_{FM} + R_{NM}} \frac{1}{R_{NM} A_{C}}
\end{equation}

\begin{equation}
    \implies J_{NM} = I_{DC} \frac{R_{FM}}{(R_{FM} + R_{NM})A_C}
\end{equation}

\begin{equation}
    \tlabel{jnmeq}
    \implies \Delta J_{NM} = \Delta I_{DC} \frac{R_{FM}}{(R_{FM} + R_{NM})A_C}
\end{equation}


Using \tref{jseq} and \tref{jnmeq} , we have

\begin{equation}
    \theta = \frac{\Delta J_{NM}}{\Delta J_{S}}
\end{equation}

\begin{equation}
    \tlabel{stfmr-spinhall}
    \implies \theta = \frac{\delta \Delta I_{H}/\delta \Delta I_{DC}}{
        \frac{2\pi f}{\lambda} \left( 
        \alpha + \frac{\sin{\phi}}{(H_{ext} + 0.5M_{eff}) \mu_{0} M_{s} t} 
        \frac{\hbar}{2e}J_{s}
        \right)
    } \frac{R_{FM}}{(R_{FM} + R_{NM})}A_C
\end{equation}

This is the final relation which we use to calculate the Spin-Hall ratio in our experiments.

\section{Second harmonic Hall}

Let us take a hall bar of NM-FM heterostructure and pass ac current ($I_{0} \sinwt $) 
through it in presence of an in-plane magnetic field ($B_0$) and measure transversal voltage 
($V_{xy}(t)$) across it. The effective magnetic field acting on the sample will have both ac
 and dc components \cite{Avci2014f}:

\begin{equation}
    \vb{B} = \vb{B_{DC}} + \vb{B_{ac}(t)}
\end{equation}

where,

\begin{equation}
    \vb{B_{DC}} = \vb{B_0} + \vb{B_{ani}}
\end{equation}

\begin{equation}
    \vb{B_{ac(t)}} = \vb{B_{AD}} + \vb{B_{FL}} + \vb{B_{Oe}} = \vb{b} \sinwt
\end{equation}

\begin{equation}
    V_{xy}(t) = R_{xy}(t) \times I_{0} \sinwt
\end{equation}

\begin{equation}
    R_{xy}(t) = R_{xy} \left( \vb{B_{DC}} \right) + \partialD{R_{xy}}{\vb{B_{ac}}} \vb{b} \sinwt
\end{equation}

\begin{equation}
    V_{xy}(t) = \left( 
        R_{xy} \left( \vb{B_{DC}} \right) +
        \partialD{R_{xy}}{\vb{B_{ac}}} \vb{b} \sinwt
    \right) I_{0} \sinwt
\end{equation}

\begin{equation}
    V_{xy}(t) = R_{xy} \left( \vb{B_{DC}} \right) I_{0} \sinwt + 
    \partialD{R_{xy}}{\vb{B_{ac}}} I_{0} \vb{b} \sinswt
\end{equation}

\begin{equation}
    V_{xy}(t) = R_{xy} \left( \vb{B_{DC}} \right) I_{0} \sinwt + 
    \partialD{R_{xy}}{\vb{B_{ac}}} I_{0} \vb{b} \left( \frac{1 - \costwt}{2} \right)
\end{equation}

\begin{equation}
    \tlabel{vxywithtrig}
    V_{xy}(t) = R_{xy} \left( \vb{B_{DC}} \right) I_{0} \sinwt + 
    \frac{1}{2} \partialD{R_{xy}}{\vb{B_{ac}}} I_{0} \vb{b}  - 
    \frac{1}{2} \partialD{R_{xy}}{\vb{B_{ac}}} I_{0} \vb{b} \costwt
\end{equation}

\begin{equation}
    \tlabel{vxywithr}
    V_{xy}(t) = R^{0}_{xy} I_{0} + R^{\omega}_{xy} I_{0} + R^{2 \omega}_{xy} I_{0}
\end{equation}

Comparing \tref{vxywithtrig} and \tref{vxywithr}, we get:

\begin{equation}
    R^{0}_{xy} = \frac{1}{2} \partialD{R_{xy}}{\vb{B_{ac}}} \vb{b} 
\end{equation}

\begin{equation}
    R^{\omega}_{xy} = R_{xy} \left( \vb{B_{DC}} \right)  \sinwt 
\end{equation}

\begin{equation}
    R^{2 \omega}_{xy}  = - \frac{1}{2} \partialD{R_{xy}}{\vb{B_{ac}}} \vb{b} \costwt
\end{equation}

This means that the first harmonic amplitude is exactly like the Dc measurement which 
will have AHE and PHE effect contributions[14]:

The second harmonic signal has three major source contributions:

\begin{enumerate}
    \item Contribution due to Oerested field
    \item Contribution due to torque on  the magnetization of FM due to the spin current 
    from the NM
    \item Contribution due to the thermal gradient along the substrate to the top layer of
     the film
\end{enumerate}

The second harmonic expression for transverse resistance can be written as:

\begin{eqnarray} \nonumber
    % \begin{split}
        R^{2 \omega}_{xy}  &=& \left( R_{AHE} - 2 R_{PHE} \cos{\theta} \sin{2 \phi} \right)
        \partialD{\cos{\theta}}{\theta_{B}} \frac{B^{\theta}_{ac}}{B_{0} \cos ({\theta_{B} - \theta})} \\
        && \nonumber + R_{PHE} \sin ^{2} {\theta} \partialD{\sin {2\phi}}{\phi_{B}} 
        \frac{B^{\phi}_{ac}}{\sin {\theta_{B}}  \sin ({\theta_{B} - \theta}) B_{0}} \\
        && + \alpha \nabla {T} I_{0} \sin{\theta}\cos{\phi}
    % \end{split}
\end{eqnarray}

When the field is applied in-plane of the sample $\theta_{B}=\pi/2$ and for permalloy with PMA, 
$\theta \approx \pi/2$, and $\phi_{B} \approx \phi$  above equation can be simplified to:


\begin{equation}
    \tlabel{r2wxy-v1}
    R^{2 \omega}_{xy}  = R_{AHE} \partialD{\cos{\theta}}
    {\theta_B} \frac{B^{\theta}_{ac}}{B_{0}} + R_{PHE} 
    \partialD{\sin{2\phi}}{\phi_B} \frac{B^{\theta}_{ac}}{B_{0}}  + 
    \alpha \nabla {T} I_{0} \sin{\theta} \cos{\phi}
\end{equation}

Calculating the derivatives at the above mentioned angles, we get,

\begin{equation}
    \tlabel{rahe}
    R_{AHE} \partialD{\cos{\theta}} {\theta_B} = R_{AHE} \sin{\theta} \partial{\theta}{\theta_{B}} = R_{AHE} \\
    \end{equation}

\begin{equation}
    \tlabel{rphe}
    R_{PHE} \partialD{\sin{2\phi}}{\phi_B} = R_{PHE} 2 \cos{2 \phi} =
    R_{PHE} \left( \cos^{2}{\phi} - 1 \right) 
\end{equation}

Substituting these values from \tref{rahe} and \tref{rphe} into \tref{r2wxy-v1} 
we get,

\begin{equation}
    \tlabel{r2wxy-v2}
    R^{2 \omega}_{xy}  = R_{AHE} \frac{B^{\theta}_{ac}}{B_{0}} 
    + R_{PHE} \left( \cos^{2}{\phi} - 1 \right) \frac{B^{\phi}_{ac}}{B_{0}}
    + \alpha \nabla {T} I_{0} \sin{\theta} \cos{\phi}
\end{equation}

The antidumping and field like torques are given by:

\begin{equation}
    \tlabel{bad}
    \vb{B_{AD}} = B_{AD} \left( \vb{m \times y} \right)
    = B_{AD} \cos{\phi}\ \hat{\theta}
\end{equation}

\begin{equation}
    \tlabel{bfloe}
    \vb{B_{FL+Oe}} = \left( B_{FL} + B_{Oe} \right) \left( \vb{m \times m \times y} \right)
    = \left( B_{FL} + B_{Oe} \right) \cos {\phi}\ \hat{\phi}
\end{equation}
Substituting these values from \tref{bad} and \tref{bfloe} in the expression 
for second harmonic resistance in \tref{r2wxy-v2} we get,
\begin{equation}
    \tlabel{r2wxy-v3}
    R^{2 \omega}_{xy}  = R_{AHE} \frac{B_{AD} \cos{\phi}\ }{B_{0}} 
    + R_{PHE} \left( \cos^{2}{\phi} - 1 \right) \frac{ \left( B_{FL} + B_{Oe} \right) }{B_{0}}
    + \alpha \nabla {T} I_{0} \sin{\theta} \cos{\phi}
\end{equation}

\begin{equation}
    \tlabel{r2wxy-v4}
    R^{2 \omega}_{xy}  = \left[ \left( R_{AHE} \frac{B_{AD}}{B_{0}} + 
    \alpha \nabla {T} I_{0} \right) \cos{\phi} \\
    + 2 R_{PHE} \left( \cos^{3}{\phi} - \cos{\phi} \right) 
    \frac{ \left( B_{FL} + B_{Oe} \right) }{B_{0}}  \right]    
\end{equation}
The simulation of the above equation gives following result.

The raw data is symmetrized before fitting to get rid of unwanted contrinutions 
due to misalignement of Hall branches, or misalignment of sample with respect 
to external fields. The odd part of the second harmonic signal and the even
part of the first harmonic signal is taken before fitting the data.
