\addchapter{Spintronics and Quantum Materials}
\label{spintronics-and-quantum-materials}

\section{Spin current - Two current model}
At a very basic level, spin current if net transfer of spins from one point to another. 
Spin Current is actually defined as a tensor $q_ij$, where $i$ denotes the direction of 
flow of spin current and j denotes the which component of spin is flowing[7]. The indices 
$i$ and $j$ are the directions in the 3D space. We can, in principle, have an net spin current
without any flow of charge current, if equal amount of up and down electron are moving in 
opposite direction. That is why these are often touted as dissipationless[8]. But in reality, 
one always needs some amount of charge current to create the spin current in the first place 
which should be taken into consideration for making a fair comparison.

There can be a net spin current if there is an imbalance between the up spin and down spin 
flowing in the opposite direction to each other. Let us define electrical spin polarization 
$P_j$ in the direction j as follows:
\begin{equation}
    \centering
    \tlabel{spin-pol-def}
    P_j=\frac{n^{\uparrow}-n^{\downarrow}}{n^{\uparrow}+n^{\downarrow}}
\end{equation}

\tref{spin-pol-def}


\section{How to create spin current?}

We can create spin current using many methods. One of the most simple methods is to take 
a ferromagnet and pass current through it and inject the current coming out of the 
ferromagnet into wherever we might need[9]. But this is a fairly inefficient method. 
There are other methods like spin pumping, spin Hall effect and inverse spin galvanic 
effect (Rashba-Edelstein effect) etc. Let us discuss these methods in detail.

\section{How to measure spin current?}


\section{Landu-Lifshitz-Gilbert Equation}
The magnetization of a sample experiences a torque perpendicular to the direction of 
applied field and the direction of magnetization. In absence of any damping the magnetization 
can keep precessing around the applied field forever. But in real materials the radius of 
precession keeps decreasing and eventually all the magnetizations point towards the applied 
magnetic field.  

\begin{equation}
    \tlabel{llg}
    \mdot = -\gamma(\mbold \times \hbold)+ \frac{\alpha}{M} \left(\mbold \times \mdot)\right)
\end{equation}

\section{Spin-Torque Ferrmomagnetic Resonance - STFMR}
When a dc magnetic field $H_{ext}$ is sweeped in the STFMR configuration, the mixing 
voltage [10], can be expressed as in \tref{llg}:

\begin{equation}
    \tlabel{vmix}
    V_{mix} = V_{0} + V_{sym} \lorsym + V_{as} \lorasym
\end{equation}

We fit this equation to extract fitting parameters \vo, \hres, \vsym, \vas, \hres, and $\Delta$. 
When there is no DC current passing through the metal layer, the resonance field can be used to 
fit effective magnetization using Kittel formula [11] using \tref{vmix}.

\begin{equation}
    \tlabel{kittel}
    f_{res} = \frac{\gamma}{4\pi} \sqrt{(H_{res} + H_{0})(H_{res} + H_{0} + M_{eff})}
\end{equation}

The Gilbert damping can be extracted from the linewidth of the resonance peak [11] using equation \tref{gilbert}.

\begin{equation}
    \tlabel{gilbert}
    \Delta H = \Delta H_{res} + \frac{2\pi f}{\lambda}\alpha
\end{equation}

When a dc current is passed through the normal metal it pumps spin current into the ferromagnetic 
layer. In presence of a spin current the equation (4) is modified as [12], [13]:

\begin{equation}
    \Delta H = \Delta H_{res} + \frac{2\pi f}{\lambda} \left( 
        \alpha + \frac{\sin{\phi}}{(H_{ext} + 0.5M_{eff}) \mu_{0} M_{s} t}  
        \frac{\hbar}{2e}J_{s}
    \right)
\end{equation}

The change in linewidth ($\delta \Delta H$) for two different values of spin current 
($\Delta J_{s} = J_{s1} - J_{s2}$) will be:

\begin{equation}
    \tlabel{deltaDeltaH}
    \delta \Delta H = \frac{2\pi f}{\lambda} \left( 
        \alpha + \frac{\sin{\phi}}{(H_{ext} + 0.5M_{eff}) \mu_{0} M_{s} t} 
        \frac{\hbar}{2e}J_{s}
        \right)
\end{equation}

The \tref{deltaDeltaH} can be rearranged to get an expression of $\Delta J_{s}$:

\begin{equation}
    \tlabel{jseq}
    \Delta J_{s} = \frac{\delta \Delta H}{
        \frac{2\pi f}{\lambda} \left( 
        \alpha + \frac{\sin{\phi}}{(H_{ext} + 0.5M_{eff}) \mu_{0} M_{s} t} 
        \frac{\hbar}{2e}J_{s}
        \right)
    }
\end{equation}

We can assume a simple model of parallel connection between the ferromagnetic layer and 
the normal metal layer to find out the current distribution within the bilayer.

\begin{figure}
   
    \centering
    
    \ig{0.3}{parallel_fm_nm.png}
    \tlabel{parallel_fm_nm}

    \caption{Parallel connection between the ferromagnetic and normal metal layer}
    
\end{figure}

The current density through the metal layer is calculated as follows:
\begin{equation}
        J_{NM} = \frac{I_{NM}}{A_{c}} = \frac{V_{DC}/R_{NM}}{A_{C}} 
        = I_{DC} \frac{R_{FM} R_{NM}}{R_{FM} + R_{NM}} \frac{1}{R_{NM} A_{C}}
\end{equation}

\begin{equation}
    \implies J_{NM} = I_{DC} \frac{R_{FM}}{(R_{FM} + R_{NM})A_C}
\end{equation}

\begin{equation}
    \tlabel{jnmeq}
    \implies \Delta J_{NM} = \Delta I_{DC} \frac{R_{FM}}{(R_{FM} + R_{NM})A_C}
\end{equation}


Using \tref{jseq} and \tref{jnmeq} , we have

\begin{equation}
    \theta = \frac{\Delta J_{NM}}{\Delta J_{S}}
\end{equation}

\begin{equation}
    \tlabel{stfmr-spinhall}
    \implies \theta = \frac{\delta \Delta I_{H}/\delta \Delta I_{DC}}{
        \frac{2\pi f}{\lambda} \left( 
        \alpha + \frac{\sin{\phi}}{(H_{ext} + 0.5M_{eff}) \mu_{0} M_{s} t} 
        \frac{\hbar}{2e}J_{s}
        \right)
    } \frac{R_{FM}}{(R_{FM} + R_{NM})}A_C
\end{equation}

This is the final relation which we use to calculate the Spin-Hall ratio in our experiments.

\section{Second Harmonic Hall}

Let us take a hall bar of NM-FM heterostructure and pass ac current ($I_{0} \sinwt $) 
through it in presence of an in-plane magnetic field ($B_0$) and measure transversal voltage 
($V_{xy}(t)$) across it. The effective magnetic field acting on the sample will have both ac
 and dc components:

\begin{equation}
    \vb{B} = \vb{B_{DC}} + \vb{B_{ac}(t)}
\end{equation}

where,

\begin{equation}
    \vb{B_{DC}} = \vb{B_0} + \vb{B_{ani}}
\end{equation}

\begin{equation}
    \vb{B_{ac(t)}} = \vb{B_{AD}} + \vb{B_{FL}} + \vb{B_{Oe}} = \vb{b} \sinwt
\end{equation}

\begin{equation}
    V_{xy}(t) = R_{xy}(t) \times I_{0} \sinwt
\end{equation}

\begin{equation}
    R_{xy}(t) = R_{xy} \left( \vb{B_{DC}} \right) + \partialD{R_{xy}}{\vb{B_{ac}}} \vb{b} \sinwt
\end{equation}

\begin{equation}
    V_{xy}(t) = \left( 
        R_{xy} \left( \vb{B_{DC}} \right) +
        \partialD{R_{xy}}{\vb{B_{ac}}} \vb{b} \sinwt
    \right) I_{0} \sinwt
\end{equation}

\begin{equation}
    V_{xy}(t) = R_{xy} \left( \vb{B_{DC}} \right) I_{0} \sinwt + 
    \partialD{R_{xy}}{\vb{B_{ac}}} I_{0} \vb{b} \sinswt
\end{equation}

\begin{equation}
    V_{xy}(t) = R_{xy} \left( \vb{B_{DC}} \right) I_{0} \sinwt + 
    \partialD{R_{xy}}{\vb{B_{ac}}} I_{0} \vb{b} \left( \frac{1 - \costwt}{2} \right)
\end{equation}

\begin{equation}
    \tlabel{vxywithtrig}
    V_{xy}(t) = R_{xy} \left( \vb{B_{DC}} \right) I_{0} \sinwt + 
    \frac{1}{2} \partialD{R_{xy}}{\vb{B_{ac}}} I_{0} \vb{b}  - 
    \frac{1}{2} \partialD{R_{xy}}{\vb{B_{ac}}} I_{0} \vb{b} \costwt
\end{equation}

\begin{equation}
    \tlabel{vxywithr}
    V_{xy}(t) = R^{0}_{xy} I_{0} + R^{\omega}_{xy} I_{0} + R^{2 \omega}_{xy} I_{0}
\end{equation}

Comparing \tref{vxywithtrig} and \tref{vxywithr}, we get:

\begin{equation}
    R^{0}_{xy} = \frac{1}{2} \partialD{R_{xy}}{\vb{B_{ac}}} \vb{b} 
\end{equation}

\begin{equation}
    R^{\omega}_{xy} = R_{xy} \left( \vb{B_{DC}} \right)  \sinwt 
\end{equation}

\begin{equation}
    R^{2 \omega}_{xy}  = - \frac{1}{2} \partialD{R_{xy}}{\vb{B_{ac}}} \vb{b} \costwt
\end{equation}

This means that the first harmonic amplitude is exactly like the Dc measurement which 
will have AHE and PHE effect contributions[14]:

The second harmonic signal has three major source contributions:

\begin{enumerate}
    \item Contribution due to Oerested Field
    \item Contribution due to torque on  the magnetization of FM due to the spin current 
    from the NM
    \item ontribution due to the thermal gradient along the substrate to the top layer of
     the film
\end{enumerate}

The second harmonic expression for transverse resistance can be written as:

\begin{eqnarray} \nonumber
    % \begin{split}
        R^{2 \omega}_{xy}  &=& \left( R_{AHE} - 2 R_{PHE} \cos{\theta} \sin{2 \phi} \right)
        \partialD{\cos{\theta}}{\theta_{B}} \frac{B^{\theta}_{ac}}{B_{0} \cos ({\theta_{B} - \theta})} \\
        && \nonumber + R_{PHE} \sin ^{2} {\theta} \partialD{\sin {2\phi}}{\phi_{B}} 
        \frac{B^{\phi}_{ac}}{\sin {\theta_{B}}  \sin ({\theta_{B} - \theta}) B_{0}} \\
        && + \alpha \nabla {T} I_{0} \sin{\theta}\cos{\phi}
    % \end{split}
\end{eqnarray}

When the field is applied in-plane of the sample $\theta_{B}=\pi/2$ and for permalloy with PMA, 
$\theta \approx \pi/2$, and $\phi_{B} \approx \phi$  above equation can be simplified to:


\begin{equation}
    \tlabel{r2wxy-v1}
    R^{2 \omega}_{xy}  = R_{AHE} \partialD{\cos{\theta}}
    {\theta_B} \frac{B^{\theta}_{ac}}{B_{0}} + R_{PHE} 
    \partialD{\sin{2\phi}}{\phi_B} \frac{B^{\theta}_{ac}}{B_{0}}  + 
    \alpha \nabla {T} I_{0} \sin{\theta} \cos{\phi}
\end{equation}

Calculating the derivatives at the above mentioned angles, we get,

\begin{equation}
    \tlabel{rahe}
    R_{AHE} \partialD{\cos{\theta}} {\theta_B} = R_{AHE} \sin{\theta} \partial{\theta}{\theta_{B}} = R_{AHE} \\
    \end{equation}

\begin{equation}
    \tlabel{rphe}
    R_{PHE} \partialD{\sin{2\phi}}{\phi_B} = R_{PHE} 2 \cos{2 \phi} =
    R_{PHE} \left( \cos^{2}{\phi} - 1 \right) 
\end{equation}

Substituting these values from \tref{rahe} and \tref{rphe} into \tref{r2wxy-v1} 
we get,

\begin{equation}
    \tlabel{r2wxy-v2}
    R^{2 \omega}_{xy}  = R_{AHE} \frac{B^{\theta}_{ac}}{B_{0}} 
    + R_{PHE} \left( \cos^{2}{\phi} - 1 \right) \frac{B^{\phi}_{ac}}{B_{0}}
    + \alpha \nabla {T} I_{0} \sin{\theta} \cos{\phi}
\end{equation}

The antidumping and field like torques are given by:

\begin{equation}
    \tlabel{bad}
    \vb{B_{AD}} = B_{AD} \left( \vb{m \times y} \right)
    = B_{AD} \cos{\phi}\ \hat{\theta}
\end{equation}

\begin{equation}
    \tlabel{bfloe}
    \vb{B_{FL+Oe}} = \left( B_{FL} + B_{Oe} \right) \left( \vb{m \times m \times y} \right)
    = \left( B_{FL} + B_{Oe} \right) \cos {\phi}\ \hat{\phi}
\end{equation}
Substituting these values from \tref{bad} and \tref{bfloe} in the expression 
for second harmonic resistance in \tref{r2wxy-v2} we get,
\begin{equation}
    \tlabel{r2wxy-v3}
    R^{2 \omega}_{xy}  = R_{AHE} \frac{B_{AD} \cos{\phi}\ }{B_{0}} 
    + R_{PHE} \left( \cos^{2}{\phi} - 1 \right) \frac{ \left( B_{FL} + B_{Oe} \right) }{B_{0}}
    + \alpha \nabla {T} I_{0} \sin{\theta} \cos{\phi}
\end{equation}

\begin{equation}
    \tlabel{r2wxy-v4}
    R^{2 \omega}_{xy}  = \left[ \left( R_{AHE} \frac{B_{AD}}{B_{0}} + 
    \alpha \nabla {T} I_{0} \right) \cos{\phi} \\
    + 2 R_{PHE} \left( \cos^{3}{\phi} - \cos{\phi} \right) 
    \frac{ \left( B_{FL} + B_{Oe} \right) }{B_{0}}  \right]    
\end{equation}
The simulation of the above equation gives following result.

The raw data is symmetrized before fitting to get rid of unwanted contrinutions 
due to misalignement of Hall branches, or misalignment of sample with respect 
to external fields. The odd part of the second harmonic signal and the even
part of the first harmonic signal is take before fitting the data.
