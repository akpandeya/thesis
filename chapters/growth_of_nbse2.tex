\addchapter{Growth and Characterization of \nbse}
\label{growth_of_nbse2}

\nbse samples in this study were  grown using Molecular beam epitaxy (MBE) on c-plane 
of sapphire substrate from Crystec. The substrate was cleaned using modified RCA method 
as described in \tref{cleaning-al2o3} and loaded in the selenide chamber of \papaya\ 
where it was degassed at \temperature{700} for about 1 hour. Nb and Se are co evaoparoated 
from e-beam and K-cells respectively such that ther flx ratio is 1:20. The first 3 monolayers 
(ML) were grown at \temperature{570} and all the subsequent layers were grown at \temperature{650}. 
The deposited sample was annealed at \temperature{700} for the first time after deposition of 3 ML 
and then intermittently after every 10-15 MLs. The growth is monitored continuously using RHEED and 
the stoichiometry and morphology is assessed intermittently using XPS and STM respectively. Once the 
sample is taken out of the chamber its structure is assessed using XRD and Raman.


\section{Cleaning \alumina substrate}
\tlabel{cleaning-al2o3}

The substrates which we recieve from the suppliers usually have both metallic and organic impurities on them.
The AFM of the substrate as recieved from the supplier is shown in \tref{afm-rec-al2o3} The recipe for cleaning 
the substrate is as follows:

\begin{figure}
   
    \centering
    \subfloat[]{
        \ig{0.31}{afm/al2o3_as_recieved.png}
        \tlabel{afm-rec-al2o3}
    }
    \subfloat[]{
        \ig{0.31}{afm/al2o3_after_cleaning.png}
        \tlabel{afm-clean-al2o3}
    }
    \subfloat[]{
        \ig{0.31}{afm/al2o3_after_annealing.png}
        \tlabel{afm-anneal-al2o3}
    }
    \caption{
        AFM image of Al2O3 sunstrates 
        \sfA~ as received form Crystec, 
        \sfB~  after modified RCA cleaning of the substrate, and 
        \sfC~  after annealing of the cleaned before loading in growth chamber. 
        The chemical cleaning helps in removing all the impurities on the substrate.
    }
    \tlabel{fig:afm-al2o3}
\end{figure}

\begin{enumerate}
    \item Soak the substrate in ethanol for 12 hours.
    \item Clean the substrate in ultrasonic bath of acetone and Isopropanol for 5 minutes each to remove 
    bulky dissolvable contaminants and particulates materials.
    \item Rinse in deionized (DI) water thoroughly and blow with dry nitrogen gas.
    \item Heat the substrate in \ammonia\ : \peroxide\ :  \water\ = 1 : 1 : 5 to about \temperature{80} for 10 minutes \ammoniumion in the 
    cleaning solution will complex with heavy metal on the substrate surface to form a soluble metal salts which is removed.
    \item Rinse in DI water thoroughly and dry with pure nitrogen gas.
    \item Soak the substrate in solution of \hcl\ : \peroxide\ : \water\ = 1: 1: 3 at about \temperature{80} for 10 minutes \hydrogenion\ ion 
    in the cleaning solution will replace with the light metal impurities to form soluble salt and be removed.
    \item Rinse in DI water thoroughly and dry with pure nitrogen gas.
    \item Clean in \sulphuricacid\ : \phosphoricacid\ = 1 : 3 at about \temperature{80} for 10min to move the oxide layer on the sapphire substrate
    \item Rinse in DI water thoroughly and dry with pure nitrogen gas.
    \item After drying the sapphire substrate anneal at \temperature{1200} for 4h preferably, but not necessarily, in the \oxygen\ atmosphere.
\end{enumerate}



We are able to get rid of the impurities after chemically cleaning it as can be seen in \tref{afm-clean-al2o3}.
The terraces of the substrate are visible in \tref{fig:afm-al2o3} after annealing it in \oxygen\ atmosphere.
The width of the terraces is dependent on the miscut angle of the substrate. Lower the miscut angle, 
wider the terraces. The cleaned substrate was then loaded into the selenide chamber of PAPAYA where 
it was first degassed at \temperature{700} for about 1 hour. The substrate surface is now ready for growth of \nbse. \\

\section{Crystallinity of \nbse}

\begin{figure}
    \centering
    \subfloat[]{
        \ig{0.25}{rheed/NbSe2_570_deg_c.png}
        \tlabel{rheed-nbse-570}
    }
    \subfloat[]{
        \ig{0.25}{rheed/NbSe2_650_deg_c.png}
        \tlabel{rheed-NbSe2-650}
    }\\
    \subfloat[]{
        \ig{0.31}{rheed/NbSe2_570_cont_deg_c.png}
        \tlabel{rheed-NbSe2-570-cont}
    }
    \subfloat[]{
        \ig{0.31}{rheed/NbSe2_570_650_deg_c.png}
        \tlabel{rheed-NbSe2-570-650}
    }
    \caption{
        RHEEDin \dirx\ direction of \nbse\
        \sfA~ when growing interfacial layer at \temperature{570}, 
        \sfB~  when growing interfacial layer at \temperature{650},  
        \sfC~  when continuing growth at \temperature{570} for > 2 h, and
        \sfD~ when growing interfacial layer at \temperature{570} and
        continuing the gorwth at \temperature{650}.
    }
    \tlabel{rheed-NbSe2-optimization}
\end{figure}

\begin{figure}
    \centering
    \ig{0.4}{rheed/NbSe2_growth_algorithm.png}
    \tlabel{rheed-NbSe2-growth-algo}
    \caption{
        Algorithm for growth of \nbse\
    } 
\end{figure}

The growth of any thin film using MBE can be affected by two major factors:
\begin{enumerate}
    \item temperature of the substrate and
    \item flux ratio of the elements of the film
\end{enumerate}
I will discuss their effect on growth of NbSe2 film one by one.

99.999\% pure Nb was evaporated using an e-beam cell and Se was evaporated using 
K-cell with a flux ratio >1:20. The interfacial layer of \nbse grrow best around 
\tempError{570}{20}. When the temperature < \temperature{500}, we get polycrystal
sample and when the temperature > \temperature{650} the island growth starts. But
the subsequent layers do not grow well at the same temperature. In order to avoid,
stranski-Krostanov growth all the subsequent layers must be grown at \tempError{650}{20}.
The first ~3 monolayer (ML) is grown at \tempError{570}{20} and all the subsequent layers 
are grown at \tempError{650}{20} \tref{rheed-NbSe2-growth-algo} and 
\tref{rheed-NbSe2-optimization}. The deposited sample is annealed at \tempError{700}{20} 
for the first time after deposition of 3 ML and then intermittently after every 10-15 ML.

\begin{figure}
   
    \centering
    \subfloat[]{
        \ig{0.31}{rheed/Al2O3_10-10.png}
        \tlabel{rheed-al2o3-10-10}
    }
    \subfloat[]{
        \ig{0.31}{rheed/NbSe2_begin-10-10.png}
        \tlabel{rheed-NbSe2-begin-10-10}
    }
    \subfloat[]{
        \ig{0.31}{rheed/NbSe2_10-10.png}
        \tlabel{rheed-NbSe2-10-10}
    }\\
    \subfloat[]{
        \ig{0.31}{rheed/Al2O3_11-20.png}
        \tlabel{rheed-al2o3-11-20}
    }
    \subfloat[]{
        \ig{0.31}{rheed/NbSe2_begin_11-20.png}
        \tlabel{rheed-NbSe2-begin-11-20}
    }
    \subfloat[]{
        \ig{0.31}{rheed/NbSe2_11-20.png}
        \tlabel{rheed-NbSe2-11-20}
    }
    \caption{
        The evolution of RHEED image of \dirx\ direction of 
        \sfA~ substrate:  \alumina, 
        \sfB~  interfacial layer of \nbse\ and,  
        \sfC~  \nbse\ after 6h growth. After rotating the sample by 30°,
         we observe \diry\ direction of 
        \sfD~ substrate ,
        \sfE~ interfacial \nbse ,
        \sfF~ \nbse\ after 6h growth.
    }
    \tlabel{rheed-NbSe2}
\end{figure}
Long streaky RHEED patterns of the (0001) Al2O3 in \dirx\ and \diry\ directions were observed 
indicating a flat and crystalline substrate surface  (\tref{rheed-al2o3-10-10} and 
\tref{rheed-al2o3-11-20}).  As the growth of forst monolayer completes the RHEED appears a 
bit diffused(\tref{rheed-NbSe2-begin-10-10} and \tref{rheed-NbSe2-begin-11-20}) but as 
the growth progresses the patter becomse streakier (\tref{rheed-NbSe2-10-10} and 
\tref{rheed-NbSe2-11-20}).It means that the grain size increases as the film grows thicker  
because of increased coalescence. The in-plane lattice parameter for this film 
is estimated to be \aerr{3.46}{0.03} using RHEED.

As mentioned above the flux of the Nb is many times higher than that of Se; it means that
the rate of growth is determined entirely the Nb flux. If we increase the flux of Nb to more
than 40 nA, the growth enters into island growth mode very quickly. This is because that 
the molecuar species do not have enough time to diffuse on the surface of the sample. This
flux corresponds to growth rate of 10 MLs/hour. This means that achieving thicker samples
is a bit challenging.

\begin{figure}
   
    \centering
    \subfloat[]{
        \ig{0.31}{xrd/nbse2-xrd.png}
        \tlabel{xrd-nbse2-full}
    }
    \subfloat[]{
        \ig{0.48}{xrd/nbse2-xrd-peak.pdf}
        \tlabel{xrd-nbse-peak}
    }
    \caption{
        X-Ray Difrraction of \nbse
        \sfA~ full range, 
        \sfB~  (002) peak.  
    }
    \tlabel{xrd-NbSe2}
\end{figure}

Once the sample is taken out from the chamber, its crystallinity is assesed using XRD.
We find no spurious peaks between 20 \degree\ and 90 \degree\ range. The standard $\theta-2\theta$ X-Ray 
diffraction (XRD) scan shows that the NbSe2 film is oriented along (0001) plane with out-of-plane 
lattice parameter as \aerr{12.77}{0.16} (see \tref{xrd-NbSe2}).

\begin{figure}
    \centering
    \ig{0.4}{tem/NbSe2-tem.png}
    \tlabel{tem-NbSe2}
    \caption{
        TEM micrograph of \nbse\
    } 
\end{figure}

High resolution transmission electron Microscopy image (shown in \tref{tem-NbSe2}) shows that 
there is some disorder in the interfacial layer but the the sample grows thicker the 
the order in the film increases. The planes of the \nbse throughout the film thickness indicating 
high crystalline order, whereas the ones at the interface with the substrate are a bit corrugated. 

\section{Morphology and stoichiometry of \nbse}
\begin{figure}
   
    \centering
    \subfloat[]{
        \ig{0.31}{stm/NbSe2_4ML.png}
        \tlabel{stm-Nbse2-4ML}
    }
    \subfloat[]{
        \ig{0.31}{stm/NbSe2_8ML.png}
        \tlabel{stm-Nbse2-8ML}
    }
    \subfloat[]{
        \ig{0.31}{stm/NbSe2_15ML.png}
        \tlabel{stm-Nbse2-15ML}
    }\\
    \subfloat[]{
        \ig{0.4}{stm/Nbse_3d_stm.png}
        \tlabel{stm-Nbse2-3D}
    }
    \subfloat[]{
        \ig{0.4}{stm/NbSe2_step_height.png}
        \tlabel{stm-nbse2-step-height}
    }
    \caption{
        STM image of \nbse
        \sfA~ 4 ML, 
        \sfB~  8 ML,  
        \sfC~  15 ML.
        \sfD~ 3D STM image of \nbse.
        \sfE~ Step height of \nbse
    }
    \tlabel{stm-NbSe2}
\end{figure}

The STM image shows that the grain size of the film increases with increasing thickness of the film. 
As the film thickness increases from 4 ML to 8 ML to 15 ML, the grain size keeps increaing as seen in 
\tref{stm-NbSe2}. It shows that as the growth progresses the smaller domains coalesce to form larger ones.
The atomic resolution in the inset of \tref{stm-NbSe2-15ML} shows that there are some 
Se vacancies in the the film. The height scan from one grain (see \tref{stm-NbSe2}) 
to another shows \aerr{6.4}{0.1} which corresponds to half the unit-cell height of \nbse.


\begin{figure}
    \centering
    \ig{0.6}{xps/nbse2-xps.pdf}
    \tlabel{xps-NbSe2}
    \caption{
        XPS Spectra of \nbse\
    } 
\end{figure}

The sample is completely impurity free because we do not see any spurious peak in the XPS
and is perfectly stoichiometric (Nb:Se = 1:2) (see \tref{xps-NbSe2})