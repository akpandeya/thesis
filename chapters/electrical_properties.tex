\addchapter{Electrical properties of quantum materials}

\section{Electrical properties of \nbse}

\begin{figure}
    \centering
    \subfloat[]{
        \ig{0.31}{NbSe2-rheed-worst.png}
        \tlabel{nbse2-rheed-worst}
    }
    \subfloat[]{
        \ig{0.31}{NbSe2-rheed-bad.png}
        \tlabel{nbse2-rheed-bad}
    }
    \subfloat[]{
        \ig{0.31}{NbSe2-rheed-good.png}
        \tlabel{nbse2-rheed-good}
    }\\
    \subfloat[]{
        \ig{0.31}{nbse2-raman-worst.pdf}
        \tlabel{nbse2-raman-worst}
    }
    \subfloat[]{
        \ig{0.31}{nbse2-raman-bad.pdf}
        \tlabel{nbse2-raman-bad}
    }
    \subfloat[]{
        \ig{0.31}{nbse2-raman-bad.pdf}
        \tlabel{nbse2-raman-good}
    }\\
    \subfloat[]{
        \ig{0.31}{nbse2-electrical-worst.pdf}
        \tlabel{nbse2-electrical-worst}
    }
    \subfloat[]{
        \ig{0.31}{nbse2-electrical-bad.pdf}
        \tlabel{nbse2-electrical-bad}
    }
    \subfloat[]{
        \ig{0.31}{nbse2-electrical-bad.pdf}
        \tlabel{nbse2-electrical-good}
    }\\
    \caption{
        \sfA~ Calculated fermi surface of NbP.
        \sfB~ Measured fermi surface of NbP
    }
    \tlabel{NbSe2-electrical-evolution}
\end{figure}
There is a clear correlation between the crystallinity of the film and the observed 
properties (Raman linewidth, electrical transport) of the film. We observe 
$A_g^1$ (235 \si{cm^{-1}}) and $E_2g^1$ (255 \si{cm^{-1}}) peaks which correspond to out-of-plane 
and in-plane vibration of Se-Nb-Se bond. The peak width of Raman signal decreases from 
40 \si{cm^{-1}} to 10 cm-1 as the film becomes more crystalline and the sample becomes more 
and more metallic and eventually a superconducting transition could be seen at about 4 \si{K} 
in a crystalline sample as shown in \tref{NbSe2-electrical-evolution}. 
The transport in the films seems to be limited by phonon $(R T^n where n<1)$ before 
it reaches the superconducting state. There seems to be a threshold carrier concentration 
of about $2.8×10^22  cm^(-3)$ below which the superconducting transition is not found. 
Even for similar thicknesses of about 9nm, the samples grown at lower temperature have more 
carrier concentration and we can see onset of SC transition beyond the threshold carrier concentration. 
Another requirement for the film to be superconducting is that it should not enter the 3D growth mode 
even once during growth. After long annealing we can make the sample more crystalline but it does not 
become superconducting.


\begin{figure}
    \centering
    \subfloat[]{
        \ig{0.31}{NbSe2-low-Tc-low-rr.pdf}
        \tlabel{nbse2-lowTc-lowrr}
    }
    \subfloat[]{
        \ig{0.31}{NbSe2-high-Tc-high-rr.pdf}
        \tlabel{nbse2-highTc-highrr}
    }
    \subfloat[]{
        \ig{0.31}{NbSe2_carrier_conc.pdf}
        \tlabel{nbse2-carrier-concentration}
    }
    \caption{
        \sfA~ Calculated fermi surface of NbP.
        \sfB~ Measured fermi surface of NbP
    }
    \tlabel{NbSe2-electrical}
\end{figure}
The films are not fully superconducting as shown in \tref{NbSe2-electrical};
they have residual resistivity after the superconducting transition. This could 
be because of some disorder in the film, for example, the grain boundaries visible 
in the STM image of the film in Figure 2(a). Figure 3 shows the temperature 
dependent magnetoresistance near superconducting transition  of two samples with 
different thicknesses. Figure 3(a) shows resistance vs temperature scan for a 20 nm 
thick NbSe2 sample, featuring possible charge density wave bump around 30K and a 
superconducting transition around 4K. The transport in the films seems to be limited by 
phonon $(R T^n where n\approx1)$ before it reaches the superconducting state. There is a 
threshold carrier concentration of about $3×10^14  cm^(-2)$ below which the superconducting 
transition is not found (see Figure 3(c)). NbSe2 is a BCS type superconductor whose 
superconductivity depends on the electron-phonon interaction. The electron-phonon 
coupling strength for a thin film depends on the carrier density and slope of the 
R-T curve at high temperature [22]. The superconductivity of the films in this study 
seem to depend on carrier density and not on the slope of the R-T curve. Ganguli et.al. 
have shown the effect of size of grains on superconductivity[23]. For the films with smaller 
grain size the coulomb interaction is greater than the electron-phonon interaction and the 
film is not superconducting. As the grain size increases the electron-phonon interaction stays 
the same while the coulomb interaction becomes smaller. Beyond a threshold grain size, 
the electron-phonon interaction becomes greater than the coulomb interaction and film 
becomes superconducting. As shown in Fig. Sx, there is correlation between grain size 
and the thickness of the film. This could be the reason why the films of thickness greater 
than 12 nm are always superconducting (Fig Sx).

To summarize, we have demonstrated that there is a correlation between carrier 
concentration of the NbSe2 films and superconductivity. The ability to control the 
superconductivity in the same material gives the flexibility to use the same material 
for different applications.

\section{Electrical properties of Weyl semimetals}

\begin{figure}
    \centering
    \subfloat[]{
        \ig{0.31}{NbP_RT.pdf}
        \tlabel{nbp-rt}
    }
    \subfloat[]{
        \ig{0.31}{NbP_n_T.pdf}
        \tlabel{nbp-nt}
    }
    \subfloat[]{
        \ig{0.31}{NbP_mu_T.pdf}
        \tlabel{nbp-mut}
    }
    \caption{
        \sfA~ Calculated fermi surface of NbP.
        \sfB~ Measured fermi surface of NbP
    }
    \tlabel{NbP-electrical}

    
\end{figure}

The analysis of electrical measurements is a bit complicated because of a thin metalic buffer layer
of Nb. Despite all the efforts to phophorize the thin layer, there might still be some contribution
from that layer. The easiest way to get around this is to analyze for a very thick film ~ 70 \si{nm}
so that the shunting effect from the mettalic Nb layer is minimal. The resistance vs temperature
measurement of Nb(Ta)P shows a mettalic behavior as shown in \tref{nbp-rt} unlike the reports which
show supercunductivity on platlets fromed from bulk crystals using ion milling. As we have observed
from the bandstructure measurement our fermil level is shifted by -0.2 \si{eV}, we should have 
contribution from both holes and electrons. The Hall effect 
The temperature 
dependence of the resistivity (Figure 6a) shows the expected metallic behavior for NbP.
We do not observe superconductivity down to 2K, in contrast to recent reports on platelets 
formed from bulk crystals using focused ion beam milling.24 The carrier density in the as-grown 
films has been extracted by standard Hall-Effect measurements, and as already anticipated from 
the Fermi-level shift determined by photoemission ($\Delta E= -0.2eV$), both electron and hole bands will 
contribute to electronic transport.  Thus, slightly non-linear transverse resistances (Rxy) as a 
function of magnetic field have been fitted by the two-carrier model to extract the densities of 
electrons (ne) and holes (np), as well as the electron (µe) and hole (µp) mobility as a function 
of temperature, summarized in Figures 6b and 6c.  Being energetically 0.2 eV below the Weyl points, 
the density of states at EF crosses mainly hole-like bands, as shown in the bulk band structure 
calculations (Figure 6e), and thus the majority carriers are holes with a higher carrier density 
(1021 - 1022 cm-3) than in the bulk crystals (1020 cm-3).47 The resulting electron (hole) mobilities 
are close to 900 cm2/Vs (300 cm2/Vs) and feature a weak temperature dependence similar to heavily-doped 
(degenerate) semiconductors.  Moreover, a positive magnetoresistance is observed both for in-plane (B//I) 
and out-of plane (B I) magnetic fields, suggesting the absence of chiral anomaly in the as-grown films.  
This result corroborates that the chiral anomaly is highly sensitive to the location of the Fermi-energy 
(EF = -0.2 eV), since the effect of chiral charge pumping is strongly diminished away from the Weyl 
points, triggered by the contribution of non-topological bands at other locations in momentum space 
(see Figure 6e).  At high magnetic fields, a linear, non-saturating behavior of the resistivity sets 
in (Supp Fig. S6), typically observed in high-mobility compensated semimetals.  